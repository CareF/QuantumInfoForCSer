%!TEX program = xelatex
%!TEX builder = simple
% or can be latexmk texify
%!TEX option = --output-driver="xdvipdfmx -V7"
% -shell-escape -8bit % For minted package
%!TEX root = ...
\documentclass[12pt,hyperref={CJKbookmarks=true},aspectratio=169]{beamer}
\usetheme[progressbar=frametitle]{metropolis}
\metroset{block=fill}
\setbeamertemplate{bibliography item}{\insertbiblabel}
% \usepackage{mydefs}
% \renewcommand{\figurename}{Fig.}
\usepackage{booktabs}
\usepackage[scale=2]{ccicons}
\usepackage{pgfplots}
\usepgfplotslibrary{dateplot}
\usepackage{xspace}
\newcommand{\themename}{\textbf{\textsc{metropolis}}\xspace}
\usepackage[noindent]{ctex}
\usepackage{amsmath,amssymb,esint}
\allowdisplaybreaks[0]
\usepackage{bm}
\usepackage{siunitx}
\usepackage{graphicx}
\graphicspath{{figures/}}
\usepackage{array}
\usepackage{ulem}
\renewcommand{\ULthickness}{1.5pt}
\usepackage{verbatim}
\usepackage{braket}

\newcommand{\dif}{\,\mathrm d}
\newcommand\mi{\mathrm{i}}
\newcommand\e{\mathrm{e}} % 定义数学模式中常用的正体字符
\bibliographystyle{unsrt}

\titlegraphic{\hfill\includegraphics[height=1.5cm]{logo.png}}
\title{面向计算机爱好者的量子信息}
\author{吕铭}
\institute{物理系毕业狗}
\date{2016年6月11日 金枪鱼之夜}

\begin{document}

\maketitle

% \begin{frame}{目录}
%   \setbeamertemplate{section in toc}[sections numbered]
%   \tableofcontents[hideallsubsections]
% \end{frame}

\section{量子态是什么}
\begin{frame}[t]\frametitle{量子态 (量子比特) 是什么}
\begin{itemize}
	\item 经典的比特: ``$0$'' 或者 ``$1$''
	\begin{itemize}
		\item 以概率 $p$ 为 ``$0$'', $1-p$ 为 ``$1$''
	\end{itemize}
	\item 量子的比特: $\ket{\psi} = c_0\ket 0 + c_1\e^{\mi\varphi}\ket 1$
	\begin{itemize}
		\item 相位 $\varphi$ 和波
		\item $\ket 0$ 的概率 $c_0^2$, $\ket 1$ 的概率 $c_1^2$
		\begin{itemize}
			\item 归一化 $c_0^2 + c_1^2 = 1$
			\item $\ket{\psi} = \alpha\ket 0 + \beta\ket 1$
		\end{itemize}
		\item $(\ket 0 + \ket 1)/\sqrt 2$ 的概率? ------ 线性代数
		\item $\bra \psi = (\ket \psi)^\dagger$, $p = |\braket{\psi|\psi'}|^2$
		$$
		\ket\psi \equiv \begin{pmatrix}
			\alpha \\ \beta
		\end{pmatrix}
		$$
	\end{itemize}
\end{itemize}
\end{frame}

\begin{frame}[t]\frametitle{一些概念}
\begin{itemize}
	\item 纯态 $\ket \psi$: 包含全部信息 (Bell 不等式)
	\item 混合态: 经典概率与量子态合在一起的描述.. 
	通常 ``退相干'' 到 (完全) 混合态
	\item 直积态: 两个或多个比特简单放在一起的状态
	$$
	\ket{\psi} = \ket 0 \ket 0 \equiv \ket 0\otimes \ket 0
	$$
	\item Bell 态与最大纠缠态
	$$
	\ket{\Psi^\pm} = \ket 0 \ket 0 \pm \ket 1 \ket 1; \quad
	\ket{\Phi^\pm} = \ket 0 \ket 1 \pm \ket 1 \ket 0
	$$
	\item 测量与态的坍缩
\end{itemize}
\end{frame}

\begin{frame}[t]\frametitle{所以具体一点是什么}
\begin{minipage}[t]{0.6\linewidth}
\begin{itemize}
	\item $\ket 0$ 和 $\ket 1$ 具体是什么? \\
	其实大家还在探索中, 可能的答案包括: 
	\begin{enumerate}
		\item 超导电路 (现在最火的)
		\item 晶体缺陷 (NV-色芯)
		\item 离子阱 (存储)
		\item 光子 (偏振) 与非线性光学
		\item 核磁共振 (NMR)
		\item 拓扑序
		\item ......
	\end{enumerate}
\end{itemize}
\end{minipage}%
\begin{minipage}[t]{0.4\linewidth}
\begin{itemize}
	\item 骗! 经! 费!
	\item 夸大宣传
	\item 困难
	\begin{enumerate}
		\item 相干时间
		\item 可操作性
		\item 可拓展性
	\end{enumerate}
	\item 大体上说我们还在做二极管的阶段
\end{itemize}
\end{minipage}
\end{frame}

\section{量子通信与量子密钥分发}

\subsection{量子纠缠}
\begin{frame}[t]\frametitle{量子纠缠 (Entanglement)}
\begin{itemize}
	\item 超出经典相关性的量子关联 (Bell 不等式)
	\item Bell 态是最大纠缠态: \\
	合在一起是纯态, 分开看是完全混合态
	\item 没有相互作用, 不能超光速通信!!! 
	\item 对于量子传态 (teleportation) 和量子中继有重要作用
	\item 涉及基础物理的解释 (神棍... )
	\item 潜在的保密计算
\end{itemize}
\end{frame}

\begin{frame}[t]\frametitle{对于测量的限制}
\begin{itemize}
	\item 测量一定伴随干扰\\
	如果某一次测量的结果与态有关, 那么测量后态一定发生改变
	\item 不可克隆原理\\
	没有办法从 $\ket\psi$ 得到 $\ket\psi\ket\psi$
	\item 不可分辨原理\\
	如果 $\ket{\psi_1}$ 和 $\ket{\psi_2}$ 不正交, 
	那么对于单个比特没有办法完美地区分二者
	\item 密钥分发: BB84 协议 (Bennett and Brassard, 1984)
\end{itemize}
\end{frame}

\begin{frame}[t]\frametitle{量子通信/量子密钥分发}
\begin{itemize}
	\item 经典信道与量子信道共同使用
	\item 主要 (唯一) 优势是物理规律保护的安全性
	\item 最常见的是借助光纤/自由空间光子
	\item 主要困难在于空间衰减和单光子源效率
	\item 已经有产品了! (潘建伟, 济南试验网)
	\item 卫星试验中... 光纤量子通信 50km 中继
\end{itemize}
\end{frame}

\section{量子计算}
\begin{frame}[t]\frametitle{量子门 (Quantum gate)}
\begin{itemize}
	\item 描述量子比特的演化: 幺正算符 ($U U^\dagger = I$)
	% \item 可编程的量子系统: 
	% $ V\ket{\alpha}\ket{n} = (U_n\ket{\alpha})\ket n $
	\item 通用量子门集合: 
	\begin{itemize}
		\item 对于单量子比特, 任意量子门可以用 2 个基本门的组合来近似, 
		门序列长度 $\mathcal O (-\log^c \epsilon)$ ($\epsilon$ 是近似误差)
		\item 对于 $N$ 量子比特, 只需要增加对于两两产生纠缠的量子门 $W_{ij}$
	\end{itemize}
	\begin{align*}
		&H = \frac{1}{\sqrt 2}\begin{pmatrix}
		1 & 1 \\
		1 & -1
		\end{pmatrix}; \quad 
		T = \begin{pmatrix}
			\e^{-\mi\pi/2} & 0 \\
			0 & \e^{\mi\pi /8}
		\end{pmatrix} \\
		&\mathtt{CNOT} = 
		\ket 0 \bra 0 \otimes I + \ket 1 \bra 1 \otimes \mathtt{NOT} 
		= \begin{pmatrix}
			1 & 0 & 0 & 0\\
			0 & 1 & 0 & 0\\
			0 & 0 & 0 & 1\\
			0 & 0 & 1 & 0
		\end{pmatrix}
	\end{align*}
\end{itemize}
\end{frame}

\begin{frame}[t]\frametitle{量子算法}
\begin{itemize}
	\item Grover 搜索算法: 已知函数 $f:\{1, \cdots, N\}\mapsto \{0, 1\}$, 
	假定 $f(n) = 1$ 数量 $S$ 极少, 寻找这样的 $n$. 
	\begin{itemize}
		\item 时间复杂度: 
		$\mathcal O(\sqrt N)$ (经典算法 $\mathcal{O}(N)$)
		\item 从 $\ket\psi = (\sum_n \ket{n})/\sqrt{N}$ 出发, 
		经过 $\mathcal O(\sqrt{N/S})$ 个门, 以几率 $p \ge 1 - S/N$ 成功
	\end{itemize}
	已经证明, $\mathcal O(\sqrt{N})$ 是量子力学范围内能做到的最好 
	(意味着无法解决 NP 问题) \\
	\item Shor 质数分解算法: 来源于数论中质因数和取余函数的周期性的关系, 
	\begin{itemize}
		\item 时间复杂度: 
		$\mathcal O (L^3)$ (经典算法 $\mathcal O (\e^{L^{1/3}\log^{2/3} L})$)
	\end{itemize}
	\item BQP (bounded error quantum polynomial) 问题
\end{itemize}
\end{frame}

\begin{frame}[t]\frametitle{其他}
\begin{itemize}
	\item ``异端''
	\begin{itemize}
		\item 系综量子计算
		\item D-Wave: 量子退火机, 复杂度用 $\mathcal O(f(\Delta E))$ 描述
		\item 拓扑量子计算: ???
	\end{itemize}
	\item 量子纠错
	\item 如果量子计算机做不出来怎么办? 
	\item 如果量子计算机做出来怎么办? 
	\begin{itemize}
		\item 脆弱的存储, 复杂的控制
		\item 潜在稍强的计算能力, 潜在更好的功耗控制
	\end{itemize}
\end{itemize}
\end{frame}

\plain{The End...\\ Thank you for listening! \\ Q \& A?}
\end{document}
